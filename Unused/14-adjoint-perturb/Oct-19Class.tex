\documentclass[12pt]{article}
\usepackage[top=1in, bottom=1in, left=1in, right=1in]{geometry}

\usepackage{setspace}
\onehalfspacing

\usepackage{amssymb}
%% The amsthm package provides extended theorem environments
\usepackage{amsthm}
\usepackage{epsfig}
\usepackage{times}
\renewcommand{\ttdefault}{cmtt}
\usepackage{amsmath}
\usepackage{graphicx} % for graphics files

% Draw figures yourself
\usepackage{tikz} 

% writing elements
\usepackage{mhchem}

% The float package HAS to load before hyperref
\usepackage{float} % for psuedocode formatting
\usepackage{xspace}

% from Denovo Methods Manual
\usepackage{mathrsfs}
\usepackage[mathcal]{euscript}
\usepackage{color}
\usepackage{array}

\usepackage[pdftex]{hyperref}
\usepackage[parfill]{parskip}

% math syntax
\newcommand{\nth}{n\ensuremath{^{\text{th}}} }
\newcommand{\ve}[1]{\ensuremath{\mathbf{#1}}}
\newcommand{\Macro}{\ensuremath{\Sigma}}
\newcommand{\rvec}{\ensuremath{\vec{r}}}
\newcommand{\vecr}{\ensuremath{\vec{r}}}
\newcommand{\omvec}{\ensuremath{\hat{\Omega}}}
\newcommand{\vOmega}{\ensuremath{\hat{\Omega}}}
\newcommand{\sigs}{\ensuremath{\Sigma_s(\rvec,E'\rightarrow E,\omvec'\rightarrow\omvec)}}
\newcommand{\el}{\ensuremath{\ell}}
\newcommand{\sigso}{\ensuremath{\Sigma_{s,0}}}
\newcommand{\sigsi}{\ensuremath{\Sigma_{s,1}}}
%---------------------------------------------------------------------------
%---------------------------------------------------------------------------
\begin{document}
\begin{center}
{\bf NE 250, F15\\
October 19, 2015 
}
\end{center}

More on \textbf{Adjoint Transport Equation} (L\&M 1.6).

Last time we started derived the adjoint transport equation. We used the \textit{adjoint operator} identity: $\langle A^{\dagger}f, g\rangle = \langle f, Ag\rangle$ for all $f$ and $g$ is the vector space. \\
We wrote this as $A=H$, $f=\zeta^{\dagger}$, and $g=\zeta$ giving $\langle H^{\dagger} \zeta^{\dagger}, \zeta\rangle = \langle\zeta^{\dagger}, H \zeta\rangle$.

We multiplied the regular TE by $\zeta^{\dagger}$, integrated over all phase space, and manipulated the result to get $\zeta$ in front of the operator.

We combined all three terms to get the adjoint form of the transport equation:
\begin{align*}
\langle\zeta^{\dagger}, H \zeta\rangle = \int_{V} dV \int_{4\pi} d\vOmega \int_0^{\infty} dE \: &\zeta (\rvec, \vOmega, E) \times \biggl[-\vOmega \cdot \nabla \zeta^{\dagger}(\vec{r}, E, \vOmega)  + \Sigma_t(\rvec,E) \zeta^{\dagger}(\vec{r}, E, \vOmega) \\&- \int d\vOmega' \int dE'\: \Sigma_s(\rvec, E \rightarrow E', \vOmega \cdot \vOmega') \zeta^{\dagger}(\rvec, \vOmega', E') \biggr]
\end{align*}

We could see how to define $H^{\dagger}$ such that  $\langle H^{\dagger} \zeta^{\dagger}, \zeta\rangle = \langle\zeta^{\dagger}, H \zeta\rangle$ is satisfied.\\
We also noted the boundary condition of zero \textit{outgoing} adjoint flux.

------------------------\\
I want to revisit the idea of adjoint flux \textit{neutron importance} since it's an important one. 

What we're going to show is that the angular adjoint flux can represent a map of how every part of phase space will influence the answer we say we are looking for. You can think of it as the reverse of the transport equation 
\begin{itemize}
\item forward: $\psi$ represents how the source particles go forward and affect the rest of the problem space.
\item adjoint: $\psi^{\dagger}$ represents how the particles from a source come into the solution space and affect the answer. 
\end{itemize}

Let's think about a problem with an external source: $H \psi = q_{ex}$, where $\psi$ meets the boundary conditions. We will just look at the point source this time
%
\begin{align*}
q_{ex} &= \delta(\rvec - \rvec_0) \delta(E - E_0) \delta(\vOmega - \vOmega_0) \\
R &= \int_V dV \int dE \int_{4\pi} d\vOmega\: q_{ex} \psi(\rvec_d, \vOmega, E)%=\psi^{\dagger}(\rvec_0, \vOmega_0, E_0)
\end{align*}

Here's how we went through this (last time with a detector response as the adjoint source):
\begin{itemize}
\item we'll set up the adjoint problem with the detector response as the source
\[H^{\dagger}\psi^{\dagger} = q_{ex}^{\dagger} \equiv  \delta(\rvec - \rvec_0) \delta(E - E_0) \delta(\vOmega - \vOmega_0)\]
where $\psi^{\dagger}(\vec{r}_s, \vOmega, E) = 0$ for $\rvec$ on $\Gamma$ and $\vOmega \cdot \hat{n} > 0$.
%
\item We multiply the initial equation (forward) by $\psi^{\dagger}$ and integrate over phase space. We can do the opposite/same to the adjoint equation. This gives
\begin{align*}
\langle\psi^{\dagger}, H\psi\rangle &= \langle\psi^{\dagger}, q_{ex}\rangle \quad \text{and}\\
\langle\psi, H^{\dagger} \psi^{\dagger}\rangle &= \underbrace{\langle\psi, \delta(\rvec - \rvec_0) \delta(E - E_0) \delta(\vOmega - \vOmega_0)\rangle }_R
\end{align*}
%
\item We can subtract the two and then apply the adjoint equivalence identity to see
\begin{align*}
\langle\psi^{\dagger}, H\psi\rangle - \langle\psi, H^{\dagger} \psi^{\dagger}\rangle &= \langle\psi^{\dagger}, q_{ex}\rangle - R \\
R &= \langle\psi^{\dagger}, q_{ex}\rangle
\end{align*}
\end{itemize}
%
So what does that mean? \\
It means that the response is the phase-space integral of the adjoint-weighted source distribution. 
\[
R = \int_V dV \int dE \int_{4\pi} d\vOmega\: q_{ex} \psi(\rvec_d, \vOmega, E)=\psi^{\dagger}(\rvec_0, \vOmega_0, E_0)
\]
What does that mean? \\
The weighting of the adjoint flux directly corresponds to how influential a given source particle will be on the answer we're looking for (what we defined as the adjoint source).

The importance provides the detector response to particles produced at $(\rvec_0, \vOmega_0, E_0)$.

We also talked through how looking at the differences in the equations we can see this as well. 

------------------------\\
The Adjoint Equation in \textbf{Multiplying Media}

We will call the fission operator $G$, where
\[
G \zeta = \frac{\chi(E)}{4\pi}\int_0^{\infty} dE' \: \nu(E') \Sigma_f(E') \int_{4\pi} d\vOmega' \psi(\rvec, \vOmega', E')
\]
To get the adjoint version, we can take the same approach as with the scattering operator to get
\[
G^{\dagger} \zeta^{\dagger} = \frac{\nu(E) \Sigma_f(E)}{4\pi}\int_0^{\infty} dE' \: \chi(E') \int_{4\pi} d\vOmega' \psi^{\dagger}(\rvec, \vOmega', E')
\]
We can combine this operator with $H$ to use the techniques in the preceding section to look at subcritical systems.\\
We replace $H$ and $H^{\dagger}$ with $H-G$ and $H^{\dagger}-G^{\dagger}$, respectively.\\

To look at the \textit{eigenvalue} form of the problem:
\[
H\psi - \frac{1}{k}G\psi = 0
\]
where $\psi$ is the fundamental solution mode, $\psi > 0$ for $\rvec \in V$, $\psi(\vec{r}_s, \vOmega, E) = 0$ for $\hat{n} \cdot \vOmega < 0$ with $\vec{r}_s \in \Gamma$.  
\[
H^{\dagger}\psi^{\dagger} - \frac{1}{k^{\dagger}}G^{\dagger}\psi^{\dagger} = 0
\]
Now $\psi^{\dagger}$ is the fundamental mode, $\psi^{\dagger} > 0$ for $\rvec \in V$, $\psi^{\dagger}(\vec{r}_s, \vOmega, E) = 0$ for $\hat{n} \cdot \vOmega > 0$ with $\vec{r}_s \in \Gamma$.  

Next, compare the two equations the way we did for response: multiply the forward equation by $\psi^{\dagger}$, integrate both over volume, and subtract for the adjoint equation multiplied by $\psi$ and integrated:
\begin{align*}
\langle\psi^{\dagger}, H\psi\rangle - \langle\psi, H^{\dagger} \psi^{\dagger}\rangle &- \frac{1}{k}\langle\psi^{\dagger}, G\psi\rangle + \frac{1}{k^{\dagger}}\langle\psi, G^{\dagger}\psi^{\dagger}\rangle = 0\\
\bigl(\frac{1}{k} - \frac{1}{k^{\dagger}} \bigr)\langle\psi^{\dagger}, G\psi\rangle  &= 0 \\
k &= k^{\dagger} \quad \text{because }G\psi > 0
\end{align*}

Cool. What do we do with that? We can figure out the effect of small changes to critical systems, which is super useful in NE!

----------------------------\\
\textbf{Reactivity} of a \textbf{Perturbed System}

Recall that reactivity is the measure of a system's deviation from critical: $\rho = \frac{k-1}{k}$.\\
We are interested in how small changes (perturbations) in a system will cause reactivity to change.\\
We can use the adjoint equation to help us.

We will denote small changes with a preceding $\delta$ and unperturbed values with a subscript $0$. Thus, for small changes in $k$ we can define 
\[
\delta \rho = \frac{\delta k}{k_0^2}
\]
For an unperturbed reactor we have these fundamental mode equations:
\begin{align*}
H_0\psi_0 &- \frac{1}{k_0}G_0\psi_0 = 0\\
H_0^{\dagger}\psi_0^{\dagger} &- \frac{1}{k_0^{\dagger}}G_0^{\dagger}\psi_0^{\dagger} = 0
\end{align*}
We can represent system perturbation by adding small quantities to the operators and fluxes (very small compared to things themselves). For the forward system we get
\[
(H_0 + \delta H)(\psi_0 + \delta \psi) - \frac{1}{k_0 + \delta k}(G_0 + \delta G)(\psi_0 + \delta \psi) = 0
\]
Because perturbations are small, we can ignore second order terms. Then
\begin{align*}
H_0 \psi_0 + \delta H \psi_0 &+ H_0\delta\psi - \bigl(\frac{1}{k_0 + \delta k}\bigr)\bigl(G_0\psi_0 + \delta G \psi_0 + G_0\delta\psi \bigr) \approx 0 \\
%
\frac{1}{k_0 + \delta k} &= \frac{1}{k_0(1 + \delta k/k_0)} \approx \frac{1}{k_0}\bigl(1 - \frac{\delta k}{k_0}\bigr) \\
\\
%
-\frac{\delta k}{k_0^2}G_0\psi_0 &\approx \underbrace{\bigl(H_0 - \frac{1}{k_0}G_0 \bigr)\psi_0}_{\text{fundamental mode}=0} + \bigl(\delta H - \frac{1}{k_0}\delta G\bigr)\psi_0  + \bigl(H_0 - \frac{1}{k_0}G_0 \bigr)\delta \psi
\end{align*}
To use inner product identities, we (as before) multiply by $\psi_0^{\dagger}$ and integrate over phase space to get
\begin{equation}
-\frac{\delta k}{k_0^2} \langle \psi_0^{\dagger}, G_0\psi_0 \rangle \approx \langle \psi_0^{\dagger},\bigl(\delta H - \frac{1}{k_0}\delta G\bigr)\psi_0 \rangle + \langle \psi_0^{\dagger},\bigl(H_0 - \frac{1}{k_0}G_0 \bigr)\delta \psi \rangle
\label{eq:perturb}
\end{equation}

Next, we can use the adjoint identity (using a modified operator) and let $\zeta = \delta\psi$ and $\zeta^{\dagger} = \psi_0^{\dagger}$ we can see:
\begin{align*}
\langle\zeta^{\dagger}, H \zeta\rangle &= \langle\zeta, H^{\dagger} \zeta^{\dagger}\rangle \\
%
\langle \psi_0^{\dagger}, \bigl(H_0 - \frac{1}{k_0}G_0 \bigr)\delta \psi \rangle &= \langle \delta \psi,  \bigl(H_0^{\dagger} - \frac{1}{k_0^{\dagger}}G_0^{\dagger} \bigr)\psi_0^{\dagger} \rangle
\end{align*}
We know that the second part of the RHS is $0$ because of the definition of the adjoint equation.\\
This means the LHS is also $0$, so we can can remove it from Eqn.~\ref{eq:perturb}.

With all of this, we can now say
\[
\frac{\delta k}{k_0^2} \approx -\frac{\langle \psi_0^{\dagger},\bigl(\delta H - \frac{1}{k_0}\delta G\bigr)\psi_0 \rangle}{\langle \psi_0^{\dagger}, G_0\psi_0 \rangle}
\]
This means that if we know the two unperturbed solutions $\psi_0^{\dagger}$ and $\psi_0$ we can find the reactivity change for any perturbations.

----------------------------\\
Other \textbf{Boundary Conditions} for the adjoint operator

For reflected and periodic boundaries:
\begin{itemize}
\item Take the reflected BC configuration and unfold it to the full extent (write explicitly in each region).
\item Then, take the adjoint of this with zero outgoing flux at the boundaries.
\item Apply the same argument as in the forward case, but now we know what's going in instead of what's going out. This preserves the form of reflected and periodic BCs.
\end{itemize}

For a fixed incoming flux:
\begin{align*}
\psi(\vec{r}_s, \vOmega, E) &= \psi_{in}(\vec{r}, \vOmega, E) \quad \hat{n} \cdot \vOmega < 0 \\
\psi^{\dagger}(\vec{r}_s, \vOmega, E) &= \psi^{\dagger}_{out}(\vec{r}, \vOmega, E) \quad \hat{n} \cdot \vOmega > 0 
\end{align*}
Now, we take the inner product of the boundary conditions with the operator as the angle-normal dot product:
\begin{align*}
\int_{\Gamma} dS \int_0^{\infty} dE \bigl[&\int_{\hat{n} \cdot \vOmega < 0} d\vOmega \: \hat{n} \cdot \vOmega \psi_{in}(\vec{r}_s, \vOmega, E) \psi_{out}^{\dagger}(\vec{r}_s, \vOmega, E) +\\ 
&\int_{\hat{n} \cdot \vOmega > 0} d\vOmega \: \hat{n} \cdot \vOmega \psi_{in}(\vec{r}_s, \vOmega, E) \psi_{out}^{\dagger}(\vec{r}_s, \vOmega, E) \bigr] 
\end{align*}
We need this to vanish, thus
\begin{align*}
\int_{\Gamma} dS \int_0^{\infty} dE \int_{\hat{n} \cdot \vOmega > 0} d\vOmega \: &\bigl[-|\hat{n} \cdot \vOmega| \psi_{in}(\vec{r}_s, |\vOmega|, E) \psi_{out}^{\dagger}(\vec{r}_s, |\vOmega|, E) +\\
&\hat{n} \cdot \vOmega \psi_{in}(\vec{r}_s, \vOmega, E) \psi_{out}^{\dagger}(\vec{r}_s, \vOmega, E) \bigr] =0
\end{align*}
Some notes on this:
\begin{itemize}
\item The boundary condition  has an integral nature for $\psi_{out}^{\dagger}$.
\item This balance requires both $\psi_{out}^{\dagger}$ and $\psi_{in}$. This means we have to solve both the forward and adjoint TEs to determine these values and therefore the BCs.
\item the contribution of outgoing ``adjoint neutrons" to detector response must equal the contribution of incoming ``forward neutrons": $R = \langle \psi^{\dagger}, q_{ex} \rangle$.
\end{itemize}

\end{document}
