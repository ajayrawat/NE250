\documentclass[12pt]{article}
\usepackage[top=1in, bottom=1in, left=1in, right=1in]{geometry}

\usepackage{setspace}
\onehalfspacing

\newif\ifeqns
\eqnstrue

\usepackage{amssymb}
%% The amsthm package provides extended theorem environments
\usepackage{amsthm}
\usepackage{epsfig}
\usepackage{times}
\renewcommand{\ttdefault}{cmtt}
\usepackage{amsmath}
\usepackage{graphicx} % for graphics files

% Draw figures yourself
\usepackage{tikz} 

% writing elements
\usepackage{mhchem}

% The float package HAS to load before hyperref
\usepackage{float} % for psuedocode formatting
\usepackage{xspace}

% from Denovo Methods Manual
\usepackage{mathrsfs}
\usepackage[mathcal]{euscript}
\usepackage{color}
\usepackage{array}

\usepackage[pdftex]{hyperref}
\usepackage[parfill]{parskip}

% math syntax
\newcommand{\nth}{n\ensuremath{^{\text{th}}} }
\newcommand{\ve}[1]{\ensuremath{\mathbf{#1}}}
\newcommand{\Macro}{\ensuremath{\Sigma}}
\newcommand{\rvec}{\ensuremath{\vec{r}}}
\newcommand{\vecr}{\ensuremath{\vec{r}}}
\newcommand{\omvec}{\ensuremath{\hat{\Omega}}}
\newcommand{\vOmega}{\ensuremath{\hat{\Omega}}}
\newcommand{\sigs}{\ensuremath{\Sigma_s(\rvec,E'\rightarrow E,\omvec'\rightarrow\omvec)}}
\newcommand{\el}{\ensuremath{\ell}}
\newcommand{\sigso}{\ensuremath{\Sigma_{s,0}}}
\newcommand{\sigsi}{\ensuremath{\Sigma_{s,1}}}
\newcommand{\cc}[1]{\ensuremath{\overline{#1}}}
\newcommand{\ccm}[1]{\ensuremath{\overline{\mathbf{#1}}}}
%---------------------------------------------------------------------------
%---------------------------------------------------------------------------
\begin{document}
\begin{center}
{\bf NE 250, F17\\
October 6, 2017 
}
\end{center}

Duderstadt and Hamilton Chp.\ 5.

So far we have looked at fixed source diffusion problems and methods for solving them analytically (note: numerical solution is the focus of 155 for DE and 255 for TE). Frequently, we care about systems with fission. We'll focus on that next.

We'll take a moment to talk through what's happening with fission in reactors. Neutrons are born at ``high" energies, in the MeV range. 
\begin{itemize}
\item \textbf{Thermal}: neutrons can cause fission at fast energies in any of $^{235}$U, $^{239}$Pu, or $^{238}$U. However, it is much more likely neutrons will scatter and lose energy (be moderated) by nuclei such as H or C. As the neutrons are slowed, they pass through absorption resonances of heavy  materials, especially $^{238}$U. Neutrons could also leak out of the system during that slowing down process. In thermal systems most ($\sim 85\%-90\%$) neutrons make it to thermal energies. There they will diffuse about the core and leak or be absorbed. Some of those that are absorbed will be absorbed in fuel, and some will cause fission.

\item \textbf{Fast}: In fast reactors we try to keep the neutrons fast. Leakage will still play a role at fast energies. Resonance absorption and downscattering will be much less. 
\end{itemize}
%
It is clear that energy is quite important in the fission process. Fundamentally, we want to know the fission neutrons that are born in the energy range $[E, E+dE]$ into angle $d\vOmega$ and $\vOmega$. This gives a fission term of 
\[
\frac{\chi(E)}{4\pi} \int_0^{\infty} dE' \int_{4\pi} d\vOmega' \: \nu(E') \Sigma_f(\vec{r}, E') \psi(\vec{r}, E', \vOmega', t) \:.  
\] 
In diffusion we use the scalar flux rather than the angular flux. Further, for our current work we'll collapse to the one-group case to get an effective fission source:
\[
\langle \nu \Sigma_f (\vec{r}) \rangle = \frac{\int_0^{\infty} dE'\: \nu(E') \Sigma_f(\vec{r}, E') \phi(\vec{r}, E', t)} {\int_0^{\infty} dE'\: \phi(\vec{r}, E', t)} \equiv \nu \Sigma_f (\vec{r}) 
\]

We can write the one-group diffusion equation with fission
\begin{equation}
\frac{1}{v} \frac{\partial \phi(\vec{r}, t)}{\partial t} - \nabla \cdot D(\vec{r}) \nabla	\phi(\vec{r}, t) + \Sigma_a(\vec{r}) \phi(\vec{r}, t) = \Sigma_f (\vec{r})\phi(\vec{r}, t)
\end{equation}

\subsection*{Solutions with time dependence}
Let's start by looking at a uniform slab. We'll keep time dependence.
\begin{align*}
\frac{1}{v} \frac{\partial \phi(x, t)}{\partial t}& - D(\vec{r}) \frac{\partial ^2 \phi(x, t)}{\partial x^2} + \Sigma_a(x) \phi(x, t) = \Sigma_f (x)\phi(x, t) \\
\phi(x,0) &= \phi_0(x) = \phi_0(-x) \quad \text{symmetric}\\
\phi(\frac{\tilde{a}}{2}, t) &= \phi(-\frac{\tilde{a}}{2}, t) = 0
\end{align*}
The starting condition of symmetry (in a symmetric system) means that our flux will stay symmetric over all times.

Notice that by retaining time dependence we have a PDE instead of an ODE. The way we'll handle this here is by assuming a separation of variables:
\[
\phi(x,t) = \psi(x) T(t)
\]
again noting that $\psi$ is a function and not angular flux.

We can substitute this in and divide by $\psi(x) T(t)$ to get
\[
\frac{1}{T}\frac{dT}{dt} = \frac{v}{\psi}\bigl[ D\frac{d^2 \psi}{dx^s} + (\nu \Sigma_f - \Sigma_a) \psi(x) \bigr] = constant \equiv -\lambda \:.
\]
This has to be constant b/c one side of the equation varies in space and one varies in time. 

We now have two ODEs:
\begin{align*}
\frac{dT}{dt} &= -\lambda T(t) \\
\bigl[ D\frac{d^2 \psi}{dx^s} &+ (\nu \Sigma_f - \Sigma_a) \psi(x) = -\frac{\lambda}{v} \psi(x) \:.
\end{align*}
The time dependent solution is one we're quite used to:
\[
T(t) = T(0) e^{-\lambda t}\:,
\]
where $T(0)$ is something we must find or be given.

For the spatial problem, we will rearrange the equation adapt our original boundary conditions:
\begin{align*}
\bigl[ D\frac{d^2 \psi}{dx^s} &+ (\frac{\lambda}{v} + \nu \Sigma_f - \Sigma_a) \psi(x) = 0 \\
\psi(\frac{\tilde{a}}{2}, t) &= \psi(-\frac{\tilde{a}}{2}, t) = 0\:.
\end{align*}

We're going to use the eigenvalue approach we learned last time. Consider the problem
\begin{align*}
\frac{d^s \psi_n}{dx^2} &+ B_n^2 \psi_n(x) = 0 \:,\\
\psi_n(\frac{\tilde{a}}{2}, t) &= \psi_n(-\frac{\tilde{a}}{2}, t) = 0\:.
\end{align*}
This time we know we have symmetric solutions, so we only have a cos term:
\begin{align*}
\psi_n(x) &= \cos(B_n x) \quad \text{eigenvectors}\\
B_n^s &= (\frac{n\pi}{\tilde{a}})^2 \quad n = \text{ odd} \quad \text{eigenvalues}
\end{align*}

We can compare our eigenvalue problem to our space-dependent ODE. If we rearrange things, we can see that 
\[
\lambda = v \Sigma_a + v D B_n^2 - v \nu \Sigma_f \equiv \lambda_n \:.
\]
The $\lambda_n$ are the time eigenvalues of the equation since they go with the time behavior. 

The general solution of our problem must then look like
\[
\phi(x,t) = \sum_{n=\text{odd}} A_n e^{(-\lambda_n t)} \cos(\frac{n \pi x}{\tilde{a}}) \:.
\]
We can use the initial condition and orthogonality to find $A_n$:
\begin{align*}
\phi(x,0) &= \phi_0(x) = \sum_{n=\text{odd}} A_n \cos(\frac{n \pi x}{\tilde{a}})\\
%
A_n &= \frac{2}{\tilde{a}}\int_{-\frac{\tilde{a}}{2}}^{\frac{\tilde{a}}{2}} dx\: \phi_0(x) \cos(\frac{n \pi x}{\tilde{a}}) \:.
\end{align*}

This means the flux (for our symmetric distribution) is a superposition of modes weighted by an exponential
\[
\phi(x,t) = \sum_{n=\text{odd}} \bigl[\frac{2}{\tilde{a}}\int_{-\frac{\tilde{a}}{2}}^{\frac{\tilde{a}}{2}} dx'\: \phi_0(x') \cos(B_n x')\bigr] e^{(-\lambda_n t)} \cos(B_nx) \:,
\]
where
\[
\lambda_n = v \Sigma_a + v D B_n^2 - v \nu \Sigma_f \qquad B_n = \frac{n \pi}{\tilde{a}} \:.
\]
Note that the separation of variables approach is essentially the same as the eigenfunction expansion approach. There are times when that is the better path to take.

\subsection*{Behavior at long times}
You'll note that as time goes on, only the first eigenmodes will survive. This means
\[
\phi(x,t) = A_1 e^{(-\lambda_1 t)}\cos(B_1 x)
\]
This mode is characterized by
\[
B_1^2 = (\frac{\pi}{\tilde{a}})^2 \equiv B_g^2 \:,
\]
which we call the \textbf{geometric buckling}. This is a measure of the curvature of the shape of the dominant mode:
\[
B_1^2 = - \frac{1}{\psi_1}\frac{d^2 \psi_1}{dx^2}\:.
\]

\subsection*{Criticality}
We also care a lot about long time behavior and reactor criticality.  This is where the reactor reaches a time-independent neutron flux, in the absence of sources other than fission, because each neutron causes one more neutron (neglecting depletion of fissionable material). 

If we look at our solution, where the higher modes decay over time, we can see that the fundamental time mode needs to vanish
\begin{align*}
\phi(x,t) &= A_1 e^{(-\lambda_1 t)}\cos(B_1 x) + \sum_{n=3,\text{odd}} A_n e^{(-\lambda_n t)} \cos(B_nx)\\
%
\lambda_1 &= 0 = v(\Sigma_a - \nu \Sigma_f) + v D B_1^2\\
%
\phi(x,t) &= A_1 \cos(B_1 x) \neq f(t)
\end{align*}

We can use the time eigenvalue of zero to get another relationship, that we call the \textbf{material buckling}.
\[
\frac{\nu \Sigma_f - \Sigma_a}{D} \equiv B_m^2
\]
The material buckling only contains materials, where as geometric buckling only represents geometry. The materials and geometry must match for criticality. Otherwise, we need to adjust our materials or our geometry:
\begin{itemize}
\item $B_m^2 > B_g^2 \rightarrow \lambda_1 < 0$ gives a supercritical state
\item $B_m^2 = B_g^2 \rightarrow \lambda_1 = 0$ gives a critical state
\item $B_m^2 < B_g^2 \rightarrow \lambda_1 > 0$ gives a subcritical state
\end{itemize}
Another way to think of this is that by increasing core size we decrease $B_g^2$ (less slope, less leakage), which increases system multiplication. When we increase the concentration of fissionable material, we increase $B_m^2$, and increase system multiplication. Those physics are represented in the math we just described. 

How does all of that relate to $k$? Recall
\[
k = \frac{\int_V dV\: \nu \Sigma_f \phi(\vec{r})}{\int_V dV\: [\Sigma_a \phi(\vec{r}) - D \nabla^2 \phi(\vec{r})]} = 1 \:.
\]
From past courses, hopefully you also recall that
\[
k_{\infty} = \frac{\nu \Sigma_f}{\Sigma_a}\:.
\]
We also know $L^2 = D/\Sigma_a$. Thus we can say
\[
\nabla^2 \phi(\vec{r}) + \frac{k_{\infty} - 1}{L^2} \phi(\vec{r}) = 0\:.
\]
If we compare this to our eigenvalue equation
\[
\nabla^2 \psi + B^s \psi(\vec{r}) = 0\:,
\]
we can see that the system will only be critical when
\[
B^2 = \frac{k_{\infty} - 1}{L^2} = B_m^2
\]
\end{document}