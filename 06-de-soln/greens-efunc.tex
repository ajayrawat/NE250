\documentclass[12pt]{article}
\usepackage[top=1in, bottom=1in, left=1in, right=1in]{geometry}

\usepackage{setspace}
\onehalfspacing

\newif\ifeqns
\eqnstrue

\usepackage{amssymb}
%% The amsthm package provides extended theorem environments
\usepackage{amsthm}
\usepackage{epsfig}
\usepackage{times}
\renewcommand{\ttdefault}{cmtt}
\usepackage{amsmath}
\usepackage{graphicx} % for graphics files

% Draw figures yourself
\usepackage{tikz} 

% writing elements
\usepackage{mhchem}

% The float package HAS to load before hyperref
\usepackage{float} % for psuedocode formatting
\usepackage{xspace}

% from Denovo Methods Manual
\usepackage{mathrsfs}
\usepackage[mathcal]{euscript}
\usepackage{color}
\usepackage{array}

\usepackage[pdftex]{hyperref}
\usepackage[parfill]{parskip}

% math syntax
\newcommand{\nth}{n\ensuremath{^{\text{th}}} }
\newcommand{\ve}[1]{\ensuremath{\mathbf{#1}}}
\newcommand{\Macro}{\ensuremath{\Sigma}}
\newcommand{\rvec}{\ensuremath{\vec{r}}}
\newcommand{\vecr}{\ensuremath{\vec{r}}}
\newcommand{\omvec}{\ensuremath{\hat{\Omega}}}
\newcommand{\sigs}{\ensuremath{\Sigma_s(\rvec,E'\rightarrow E,\omvec'\rightarrow\omvec)}}
\newcommand{\el}{\ensuremath{\ell}}
\newcommand{\sigso}{\ensuremath{\Sigma_{s,0}}}
\newcommand{\sigsi}{\ensuremath{\Sigma_{s,1}}}
\newcommand{\cc}[1]{\ensuremath{\overline{#1}}}
\newcommand{\ccm}[1]{\ensuremath{\overline{\mathbf{#1}}}}
%---------------------------------------------------------------------------
%---------------------------------------------------------------------------
\begin{document}
\begin{center}
{\bf NE 250, F17\\
October 5, 2017 
}
\end{center}

Duderstadt and Hamilton Chp.\ 5.

The last thing we were working on was how to solve the \textbf{one-speed} diffusion equation:

\begin{equation*}
\frac{1}{v_1}\frac{\partial \phi_1(\rvec,t)}{\partial t} = S_1(\rvec,t) - 
\Sigma_{a,1}(\rvec)\phi_1(\rvec,t) + \nabla\cdot[D_1(\rvec)\nabla\phi_1(\rvec,t)]
\end{equation*}
We had then looked at steady state and homogeneous to get 
\ifeqns
\begin{equation*}
\nabla^2\phi(\vecr) - \frac{1}{L^2}\phi(\rvec) = -\frac{S(\rvec)}{D},
\end{equation*}
\else
\vspace*{3em}
\fi

The one-group formulation is actually quite useful when thinking through initial design concepts. You can layer together geometry components and see if ideas have any validity at all, and this is something you can do by hand to check computational results. 

There are three main ways to solve this. Last time we focused on variation of constants/parameters, which is likely what you learned first in differential equations class. You can get more information in any Diff.\ Eq.\ text book, or start at Wikipedia and go from there (\url{https://en.wikipedia.org/wiki/Variation_of_parameters}).

This class we'll look at \textbf{Green's Functions} and the \textbf{Eigenfunction Expansion} method. Having multiple tools available can allow you to solve different problems with whichever one is the easiest for that problem. Further, these methods can be used in more sophisticated contexts later on, so it's worth understanding each of them. 

\subsection*{Green's Functions}
A Green's function is the impulse response of an inhomogeneous linear differential equation defined on a domain, with specified initial conditions or boundary conditions.

Through the superposition principle for linear operator problems, the convolution of a Green's function with an arbitrary function $f(x)$ on that domain is the solution to the inhomogeneous differential equation for $f (x)$. \url{https://en.wikipedia.org/wiki/Green\%27s_function}

To see how this works, we start with two equations. We'll consider a unit source located at $\vec{r'}$. The first equation is the way we're used to seeing it:
\begin{align}
\nabla &\cdot D \nabla \phi(\vec{r}) + \Sigma_a \phi(\vec{r}) = -S(\vec{r}) \:, \qquad \forall \vec{r} \in V  \\
%
&\phi(\vec{r}) = 0\:, \qquad \forall \vec{r} \in \mathcal{S} \equiv \partial V\:.\ \nonumber
 \label{eq:de}
\end{align}
%
We will also state that $G(\vec{r}, \vec{r'})$ is the flux at $\vec{r}$ due to a unit source at $\vec{r'}$. Since this is a solution to our problem, we can write:
\begin{align}
\nabla &\cdot D \nabla G(\vec{r}, \vec{r'}) + \Sigma_a G(\vec{r}, \vec{r'}) = -\delta(\vec{r} - \vec{r'}) \:, \qquad \forall \vec{r} \in V  \\
%
&G(\vec{r}, \vec{r'}) = 0\:, \qquad \forall \vec{r} \in \mathcal{S} \equiv \partial V\:. \nonumber
\label{eq:green}
\end{align}

Now, we'll take the approach of multiplying the LHS of each equation by the opposite solution function, subtracting, and integrating over phase space. This is a common math trick that we'll use again later.
\begin{align*}
\int_V dV\: \bigl[(\nabla &\cdot D \nabla \phi(\vec{r}))G(\vec{r}, \vec{r'}) + (\Sigma_a \phi(\vec{r})) G(\vec{r}, \vec{r'}) \\
-(\nabla &\cdot D \nabla G(\vec{r}, \vec{r'}))\phi(\vec{r}) - (\Sigma_a G(\vec{r}, \vec{r'}))\phi(\vec{r}) \bigr] = 0\\
\end{align*}
We can see this by using divergence theorem and the boundary conditions to note:
\[\int_{\mathcal{S}} d\mathcal{S}\: \hat{e}_s \cdot D[\nabla \phi(\vec{r}))G(\vec{r}, \vec{r'}) - \nabla G(\vec{r}, \vec{r'}))\phi(\vec{r})] = 0\]

From there, we can subtract the right hand sides to see:
\begin{align*}
\int_V dV\: &S(\vec{r})G(\vec{r}, \vec{r'}) - \delta(\vec{r} - \vec{r'})\phi(\vec{r}) = 0\\%
\phi(\vec{r'}) = \int_V dV\: S(\vec{r})G(\vec{r}, \vec{r'})\\
\text{or }\: \phi(\vec{r}) = \int_V dV\: S(\vec{r'})G(\vec{r'}, \vec{r})
\end{align*}
%
Thus, we have shown that the Green's function is the general solution to the diffusion equation. This is also known as a kernel.

The challenge is: how do we find $G(\vec{r}, \vec{r'})$? We'll use some of the strategies we've already developed. Consider a plane source in an infinite medium:
\begin{align*}
\frac{d^2\phi}{dx^2} &- \frac{1}{L^2}\phi(x) = -S(x)\delta(x)\\
&\lim_{x \rightarrow \pm \infty} \phi(x) = 0\\
%
\frac{d^2 G(x,x')}{dx^2} &- \frac{1}{L^2}G(x,x') = -S(x)\delta(x)\\
&\lim_{x \rightarrow \pm \infty} G(x,x') = 0
\end{align*}

Now we use variation of constants like we were doing before to get
\[
G(x,x') = \frac{L}{2D} H(x-x') \exp[\frac{-|x-x'|}{L}] +  \frac{L}{2D}(1- H(x-x')) \exp[\frac{|x-x'|}{L}]\]
Then we substitute this in to get
\begin{align*}
\phi &= \int_{-\infty}^{\infty} dx'\: G(x,x')\delta(x') S\\
\phi &= \frac{SL}{2D} \exp(\frac{-|x|}{L})
\end{align*}

%-------------------------
\subsection*{Eigenfunction Expansion Method}
A powerful method for solving boundary value problems is to obtain the solution as an expansion in the set of eigenfunctions. 

\subsection*{Eigenvalue Review}

Eigenvalues are a special set of scalars associated with a linear system of equations (a matrix equation) that are sometimes also known as characteristic roots. 

Each eigenvalue is paired with a corresponding so-called eigenvector.

The decomposition of a square matrix $\ve{A}$ into eigenvalues and eigenvectors is known as eigen decomposition, and the fact that \textbf{this decomposition is always possible as long as the matrix consisting of the eigenvectors of $\ve{A}$ is square} is known as the eigen decomposition theorem.

Let $\ve{A} \in \mathbb{R}^{n \times n}$. If there is a vector $\vec{x} \in \mathbb{R}^{n}$ such that
%
\[\ve{A} \vec{x} = \lambda \vec{x}\]
%
for some scalar $\lambda$, then $\lambda$ is an eigenvalue of $\ve{A}$ with a corresponding (right) eigenvector $\vec{x}$. Note that the eigenvalue represents the ``stretching factor" in the direction of its associated eigenvector. 

(\url{http://math.stackexchange.com/questions/54176/is-there-a-geometric-meaning}\\ \url{-of-the-frobenius-norm})
 
We find eigenvalues by re-writing the stated relationship as $(\ve{A} - \lambda \ve{I})\vec{x}=0$ and solving for the $\lambda$s that make this true. The $\lambda$s are then the eigenvalues and the $\vec{x}$s that go with them are the corresponding eigenvectors. 

A linear system of equations has nontrivial solutions iff the determinant vanishes (Cramer's rule; related to the theorem for finding out if a solution exits that we talked about above), so the solutions of this equation are given by
%
\[\det(\ve{A} - \lambda \ve{I})=0 \:.\] 	
%
This equation is known as the characteristic equation of $\ve{A}$, and the left-hand side is known as the characteristic polynomial.

Since an \nth degree polynomial has $n$ roots, $\ve{A}_{n \times n}$ is guaranteed to have $n$ real and/or complex eigenvalues, some of which may be repeated: $\lambda_1, \lambda_2, \dots, \lambda_n$.

This gives $\ve{A}\vec{u}_1 = \lambda_1 \vec{u}_1$, etc.

The spectrum of eigenvalues of a matrix $\ve{A}$ are defined formally as
\[\sigma(\ve{A}) = [ \lambda \in \mathbb{C} : \det(\ve{A} - \lambda \ve{I})=0] \] 
and an eigenvalue is 
\[ \lambda \in \sigma(\ve{A})\:,\]

-----------------------------\\
We'll start with an example to see how the method works for our problems. We'll look at a similar problem as before:
\begin{align*}
\nabla^2\phi(\vec{r}) &- \Sigma_a\phi(\vec{r}) = -\frac{S(\vec{r})}{D} \:, \quad \forall \vec{r} \in V \\
& \phi(\vec{r}) &= 0\:, \quad \forall \vec{r} \in \mathcal{S}
\end{align*}
Now, we introduce the eigenfuction expansion, where the subscript $n$ indicates the $n$th mode:
\begin{align*}
\nabla^2 \psi_n (\vec{r}) &+ B_n^2 \psi_n (\vec{r}) = 0\:, \quad \forall \vec{r} \in V  \\
%
\nabla^2 &\psi_n (\vec{r}) =  -B_n^2 \psi_n (\vec{r}) \qquad \text{Note: } \ve{A} = \nabla^2\: \text{ and }\: \lambda = -B_n^2\\
\psi_n (\vec{r}) &= 0\:, \quad \forall \vec{r} \in \mathcal{S}
\end{align*}

We will use the property that eigenfunctions form a complete set. That is, any function, including our solution, can be expressed as a linear combination of eigenfunctions:
\[
\phi(\vec{r}) = \sum_{n=1}^N c_n \psi_n\:, \quad \text{where } c_n \text{ are coefficients.}
\]
We substitute this, and then our eigenvalue relationship, into the diffusion equation while also expanding the source term:
\begin{align}
\nabla^2\sum_{n=1}^N c_n \psi_n &- \Sigma_a\sum_{n=1}^N c_n \psi_n = -\frac{-}{D}\sum_{n=1}^N S_n \psi_n \nonumber \\
%
\sum_{n=1}^N c_n &(B_n^2 + \Sigma_a)\psi_n = -\frac{1}{D}\sum_{n=1}^N S_n \psi_n
\label{eq:evec}
\end{align}

Next, we're going to use the orthonormality property of eigenvectors:
\[
\int_V dV\: \psi_n \psi_m = \left\{
	\begin{array}{ll}
		1  & \mbox{if } n = m \\
		0 & \mbox{if } n \neq m
	\end{array}
	\right.
\]
We will use this to find our $c_n$ by multiplying \autoref{eq:evec} by $\psi_m$ and integrating over volume:
\begin{align*}
\int_V dV\: \sum_{n=1}^N c_n &(B_n^2 + \Sigma_a)\psi_n \psi_m = -\int_V dV\:\frac{1}{D}\sum_{n=1}^N S_n \psi_n \psi_m \\
%
c_n &(B_n^2 + \Sigma_a) = \frac{1}{D} S_n\\
c_n &= \frac{S_n / \Sigma_a}{1 + B_n^2  L^2}
\end{align*}
Noting 
\[
S_n = \int_V dV\: S(\vec{r}) \psi_n
\]
We get
\[
\phi(\vec{r}) = \int_V dV\:\sum_{n=1}^N \\frac{S(\vec{r}) / \Sigma_a}{1 + B_n^2  L^2} \psi_n \psi_n 
\]

\subsection*{Examples}
Let's solve a problem to see what this looks like in a system we're trying to solve. 

\end{document}
