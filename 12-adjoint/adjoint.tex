\documentclass[12pt]{article}
\usepackage[top=1in, bottom=1in, left=1in, right=1in]{geometry}

\usepackage{setspace}
\onehalfspacing

\usepackage{amssymb}
%% The amsthm package provides extended theorem environments
\usepackage{amsthm}
\usepackage{epsfig}
\usepackage{times}
\renewcommand{\ttdefault}{cmtt}
\usepackage{amsmath}
\usepackage{graphicx} % for graphics files

% Draw figures yourself
\usepackage{tikz} 

% writing elements
\usepackage{mhchem}

% The float package HAS to load before hyperref
\usepackage{float} % for psuedocode formatting
\usepackage{xspace}

% from Denovo Methods Manual
\usepackage{mathrsfs}
\usepackage[mathcal]{euscript}
\usepackage{color}
\usepackage{array}

\usepackage[pdftex]{hyperref}
\usepackage[parfill]{parskip}

% math syntax
\newcommand{\nth}{n\ensuremath{^{\text{th}}} }
\newcommand{\ve}[1]{\ensuremath{\mathbf{#1}}}
\newcommand{\Macro}{\ensuremath{\Sigma}}
\newcommand{\rvec}{\ensuremath{\vec{r}}}
\newcommand{\vecr}{\ensuremath{\vec{r}}}
\newcommand{\omvec}{\ensuremath{\hat{\Omega}}}
\newcommand{\vOmega}{\ensuremath{\hat{\Omega}}}
\newcommand{\sigs}{\ensuremath{\Sigma_s(\rvec,E'\rightarrow E,\omvec'\rightarrow\omvec)}}
\newcommand{\el}{\ensuremath{\ell}}
\newcommand{\sigso}{\ensuremath{\Sigma_{s,0}}}
\newcommand{\sigsi}{\ensuremath{\Sigma_{s,1}}}
%---------------------------------------------------------------------------
%---------------------------------------------------------------------------
\begin{document}
\begin{center}
{\bf NE 250, F17\\
October 26, 2017 
}
\end{center}

\subsection*{Adjoint Transport Equation} (L\&M 1.6)

The adjoint TE has two main applications: use in MC variance reduction and use in perturbation theory for eigenvalue problems (estimate changes in the neutron multiplication caused by small changes in material properties). \\
We will start with some mathematical concepts we will use.

\textit{Inner product}
\begin{itemize}
\item An inner product is a binary operation (takes 2 arguments) that combines 2 vectors into a scalar quantity.
\item in Euclidean space, which comprises all 3D vectors, the inner product is the project of one vector onto another ($\vec{v}_1 \cdot \vec{v}_2 = |v_1||v_2|\cos(\theta)$).
\item A function space comprises all functions that have a given property. e.g.\ all continuous functions; all functions with Dirichlet BCs; etc.
\item for $f \: \in$ a function space: $f$ satisfies the properties of the space
\item one way to define an inner product: $<f,g> \equiv \int_{\rho} d\rho \: f^{*}(\rho) g(\rho)$, where superscript $*$ indicates complex conjugate.
\item Inner products are linear: $<f, [ag_1 + bg_2]> = a<f, g_1> + b<f, g_2>$, where $a$ and $b$ are scalars.
\end{itemize}

The \textit{adjoint operator} obeys the following identity: $\langle A^{\dagger}f, g\rangle = \langle f, Ag\rangle$ for all $f$ and $g$ in the vector space. Superscript $\dagger$ indicates adjoint quantities.\\
Note that if $M_1^{\dagger}$ and $M_2^{\dagger}$ are both adjoints of $M_1$ and $M_2$, respectively, then $(M_1 + M_2)^{\dagger}  = M_1^{\dagger} + M_2^{\dagger}$.

We can relate the forward and adjoint eigenspaces\\
Forward: $Mf = \lambda f$ with homogeneous BCs;\\
Adjoint: $M^{\dagger}f^{\dagger} = \lambda^{\dagger} f^{\dagger}$ with homogeneous BCs.

\begin{align*}
\langle M^{\dagger}f^{\dagger}, f\rangle - \langle f^{\dagger}, Mf\rangle &= \langle\lambda^{\dagger} f^{\dagger}, f\rangle - \langle f^{\dagger} ,\lambda f\rangle \\
0 &= \langle\lambda^{\dagger} - \lambda\rangle \langle f^{\dagger}, f\rangle \quad \text{orthogonality condition}
\end{align*}
%
If $\lambda^{\dagger, *} \neq \lambda$ then $\langle f^{\dagger}, f\rangle = 0$.\\
If $\langle f^{\dagger}, f\rangle \neq 0$ then $\lambda^{\dagger, *} = \lambda \rightarrow \lambda^{\dagger} = \lambda$.\\
If $M = M^{\dagger}$, the operator is called self-adjoint. 

\subsubsection*{Adjoint Transport Equation for non-multiplying media}

Note that we're going to use $\zeta$ to indicate this could be applied to any appropriate function (though in practice we'll use $\psi$).

We will use vacuum boundary conditions: $\zeta(\vec{r}_s, \vOmega, E) = 0$, $\hat{n} \cdot \vOmega < 0$, where $\vec{r}_s$ is on the surface and $\hat{n}$ is the outward normal at $\vec{r}_s$.\\
Recall that $\zeta(\vec{r}_s, \vOmega, E)$ must be continuous in the direction of particle travel.


We will call our transport operator $H$ .\\
Our goal is to find $H^{\dagger}$ such that $\langle\zeta^{\dagger}, H \zeta\rangle = \langle\zeta, H^{\dagger} \zeta^{\dagger}\rangle$.\\


To do this, we'll start with the LHS : we will multiply the regular TE by $\zeta^{\dagger}$, integrate over all phase space, and manipulate the result to get $\zeta$ in front of the operator.
%
\[\int_{\rho} d\rho\: \zeta^{\dagger}(\rho)\biggl[ \bigl[\vOmega \cdot \nabla + \Sigma_t\bigr] \zeta(\vec{r}, E, \vOmega) - \int_{4 \pi} d\vOmega' \int_0^{\infty} dE' \: \Sigma_s(E', \vOmega' \rightarrow E, \vOmega) \zeta(\vec{r}, E', \vOmega') \biggr]\]
% 
Let's go through this one term at a time:
\begin{itemize}
\item The total interaction is a self-adjoint operator (why?), so we can simply rearrange the order: $\zeta^{\dagger} \Sigma_t \zeta \rightarrow \zeta \Sigma_t \zeta^{\dagger}$. That is
\[\int_V dV \zeta^{\dagger}(\vec{r}, E, \vOmega) \Sigma_t(E) \zeta(\vec{r}, E, \vOmega) = \int_V dV \zeta(\vec{r}, E, \vOmega) \Sigma_t(\rvec,E) \zeta^{\dagger}(\vec{r}, E, \vOmega)\]
%
\item For the streaming term, use $\vOmega \cdot \nabla = \nabla \cdot \vOmega$ and the identity (from the chain rule)
\[\nabla \cdot \bigl(\vOmega \zeta^{\dagger} \zeta \bigr) = \zeta^{\dagger}\vOmega \cdot \nabla \zeta + \zeta \vOmega \cdot \nabla \zeta^{\dagger}\:,\]
then
\[
\int_V d^3r \:\zeta^{\dagger}\vOmega \cdot \nabla \zeta + \int_V d^3r \:\zeta \vOmega \cdot \nabla \zeta^{\dagger} = \int_S d \Gamma \hat{n} \cdot \vOmega \zeta^{\dagger} \zeta
\]
from divergence theorem. 

Now, we apply the boundary condition that we originally had for $\zeta$, which causes the surface integral to vanish for $\vOmega \cdot \hat{n} < 0$.

We have a \textit{corresponding boundary condition} for $\zeta^{\dagger}$ that at any point $\rvec_s$ on the surface, $\zeta^{\dagger}(\rvec, \vOmega, E) = 0$, $\vec{r} \in \Gamma$ and $\vOmega \cdot \hat{n} \geq 0$.

This means that the surface integral has to vanish for all $\vOmega$ giving
\[
\int_V d^3r \:\zeta^{\dagger}\vOmega \cdot \nabla \zeta = -\int_V d^3r \:\zeta \vOmega \cdot \nabla \zeta^{\dagger}
\]
%
\item Finally we'll look at the scattering term. As with total interaction, we know $\zeta^{\dagger} \Sigma_s \zeta  = \zeta \Sigma_s \zeta^{\dagger}$. \\
First, we will swap $E'$ and $E$, $\vOmega'$ and $\vOmega$ (which we can do because this is an arbitrary change)
\begin{align*}
\int d\vOmega \int dE\: &\zeta^{\dagger}(\rvec, \vOmega, E) \int d\vOmega' \int dE'\: \Sigma_s(\rvec, E' \rightarrow E, \vOmega' \cdot \vOmega) \zeta(\rvec, \vOmega', E') = \\
%
&\int d\vOmega' \int dE'\: \zeta^{\dagger}(\rvec, \vOmega', E') \int d\vOmega \int dE\: \Sigma_s(\rvec, E \rightarrow E', \vOmega \cdot \vOmega') \zeta^{\dagger}(\rvec, \vOmega, E) 
\end{align*}
Then we rearrange the integration (which we can do based on the dependencies) to get something that looks like what we want
\[
\int d\vOmega \int dE\: \zeta(\rvec, \vOmega, E) \int d\vOmega' \int dE'\: \Sigma_s(\rvec, E \rightarrow E', \vOmega \cdot \vOmega') \zeta^{\dagger}(\rvec, \vOmega', E') 
\]


\end{itemize}
%
We now combine all three terms to get the adjoint form of the transport equation (!):
\begin{align*}
\langle\zeta^{\dagger}, H \zeta\rangle = \int_{V} dV \int_{4\pi} d\vOmega \int_0^{\infty} dE \: &\zeta (\rvec, \vOmega, E) \times \biggl[-\vOmega \cdot \nabla \zeta^{\dagger}(\vec{r}, E, \vOmega)  + \Sigma_t(\rvec,E) \zeta^{\dagger}(\vec{r}, E, \vOmega) \\&- \int d\vOmega' \int dE'\: \Sigma_s(\rvec, E \rightarrow E', \vOmega \cdot \vOmega') \zeta^{\dagger}(\rvec, \vOmega', E') \biggr]
\end{align*}
%
We can see that $\langle\zeta^{\dagger}, H \zeta\rangle = \langle\zeta, H^{\dagger} \zeta^{\dagger}\rangle$ will be satisfied if we define the adjoint operator as
\[
H^{\dagger} = -\vOmega \cdot \nabla  + \Sigma_t(\rvec,E) - \int d\vOmega' \int dE'\: \Sigma_s(\rvec, E \rightarrow E', \vOmega \cdot \vOmega')
\]
Where key differences are the negative in front of the streaming term and the reversal of energy direction in the scattering cross section.\\
We also need the boundary condition of zero \textit{outgoing} adjoint flux.

------------------------\\
Interpretation of adjoint flux as \textit{neutron importance}

Let's think about a problem with an external source: $H \psi = q_{ex}$, where $\psi$ meets the boundary conditions.\\
We'd like to find out the response of a detector with cross section $\Sigma_d$ and volume $V_d$ to that source: $R = V_d \int dE\: \Sigma_d(E) \int_{4\pi} d\vOmega \: \psi(\rvec_d, \vOmega, E)$. \\
\begin{itemize}
\item We'll set up the adjoint problem with the detector response as the source
\[H^{\dagger}\psi^{\dagger} = q_{ex}^{\dagger} \equiv V_d \Sigma_d \delta(\rvec - \rvec_d)\]
where $\psi^{\dagger}(\vec{r}_s, \vOmega, E) = 0$ for $\rvec$ on $\Gamma$ and $\vOmega \cdot \hat{n} > 0$.
%
\item We can multiply the initial equation (forward) by $\psi^{\dagger}$ and integrate over phase space. We can do the opposite/same to the adjoint equation. This gives
\begin{align*}
\langle\psi^{\dagger}, H\psi\rangle &= \langle\psi^{\dagger}, q_{ex}\rangle \quad \text{and}\\
\langle\psi, H^{\dagger} \psi^{\dagger}\rangle &= \underbrace{\langle\psi, V_d \Sigma_d \delta(\rvec - \rvec_d)\rangle }_R
\end{align*}
%
\item We can subtract the two and then apply the adjoint equivalence identity to see
\begin{align*}
\langle\psi^{\dagger}, H\psi\rangle - \langle\psi, H^{\dagger} \psi^{\dagger}\rangle &= \langle\psi^{\dagger}, q_{ex}\rangle - R \\
R &= \langle\psi^{\dagger}, q_{ex}\rangle
\end{align*}
\end{itemize}
%
So what does that mean? \\
It means that the detector response is the volume integral of the adjoint-weighted source distribution. \\
What does that mean? \\
The weighting of the adjoint flux directly corresponds to how influential a given source particle will be on the answer we're looking for (what we defined as the adjoint source).

This is a pretty powerful concept. \\
We can look at a delta source to clarify this a bit.
\begin{align*}
q_{ex} &= \delta(\rvec - \rvec_0) \delta(E - E_0) \delta(\vOmega \cdot \vOmega_0) \\
R &= \psi^{\dagger}(\rvec_0, \vOmega_0, E_0)
\end{align*}
It is clear here that the importance provides the detector response to particles produced at $(\rvec_0, \vOmega_0, E_0)$.

This also makes sense if we think through the differences in the equations:
\begin{itemize}
\item $-\vOmega \cdot \nabla$: by convention, $\vOmega \cdot \nabla \psi$ implies the \textit{out}flow of $\psi$ from an infinitesimal volume.\\
A neutron that exits $dV$ has a better chance to contribute to whatever you are measuring and so has a higher importance. \\
In the adjoint equation the streaming is therefore a \textit{gain} term (recall it's a loss in the forward case, so the negative makes it production in the adjoint).
%
\item Scattering term: transposed $E'$ and $E$, $\vOmega'$ and $\vOmega$. \\
A neutron with energy $E$ and direction  $\vOmega$ has importance $\psi^{\dagger}(\rvec, \vOmega, E)$. \\
If it scatters into $E'$ and $\vOmega'$ it produces neutrons whose importance is $\psi^{\dagger}(\rvec, \vOmega', E')$.\\
Thus, the contribution to $\psi^{\dagger}(\rvec, \vOmega, E)$ from scattering is:
\[
\int_0^{\infty} dE' \int_{4\pi} d\vOmega' \:\Sigma_s(\rvec, E \rightarrow E', \vOmega \cdot \vOmega')\psi^{\dagger}(\rvec, \vOmega', E')
\]
\item Boundary condition: $\psi^{\dagger}(\rvec, \vOmega, E) = 0\:, \: \hat{n}\cdot \vOmega > 0$.\\
Neutrons exiting the problem configuration cannot scatter back and so cannot contribute to the detector inside. 
\end{itemize}


------------------------\\
Let's dig more into the the idea of adjoint flux \textit{neutron importance} since it's an impactful one. 

What we're going to show is that the angular adjoint flux can represent a map of how every part of phase space will influence the answer we say we are looking for. You can think of it as the reverse of the transport equation 
\begin{itemize}
\item forward: $\psi$ represents how the source particles go forward and affect the rest of the problem space.
\item adjoint: $\psi^{\dagger}$ represents how the particles from a source come into the solution space and affect the answer. 
\end{itemize}

Let's think about a problem with an external source: $H \psi = q_{ex}$, where $\psi$ meets the boundary conditions. We will just look at the point source this time
%
\begin{align*}
q_{ex} &= \delta(\rvec - \rvec_0) \delta(E - E_0) \delta(\vOmega - \vOmega_0) \\
R &= \int_V dV \int dE \int_{4\pi} d\vOmega\: q_{ex} \psi(\rvec_d, \vOmega, E)%=\psi^{\dagger}(\rvec_0, \vOmega_0, E_0)
\end{align*}

Here's how we went through this (last time with a detector response as the adjoint source):
\begin{itemize}
\item We'll again set up the adjoint problem but with a different source
\[H^{\dagger}\psi^{\dagger} = q_{ex}^{\dagger} \equiv  \delta(\rvec - \rvec_0) \delta(E - E_0) \delta(\vOmega - \vOmega_0)\]
where $\psi^{\dagger}(\vec{r}_s, \vOmega, E) = 0$ for $\rvec$ on $\Gamma$ and $\vOmega \cdot \hat{n} > 0$.
%
\item We multiply the initial equation (forward) by $\psi^{\dagger}$ and integrate over phase space. We can do the opposite/same to the adjoint equation. This gives
\begin{align*}
\langle\psi^{\dagger}, H\psi\rangle &= \langle\psi^{\dagger}, q_{ex}\rangle \quad \text{and}\\
\langle\psi, H^{\dagger} \psi^{\dagger}\rangle &= \underbrace{\langle\psi, \delta(\rvec - \rvec_0) \delta(E - E_0) \delta(\vOmega - \vOmega_0)\rangle }_R
\end{align*}
%
\item We can subtract the two and then apply the adjoint equivalence identity to see
\begin{align*}
\langle\psi^{\dagger}, H\psi\rangle - \langle\psi, H^{\dagger} \psi^{\dagger}\rangle &= \langle\psi^{\dagger}, q_{ex}\rangle - R \\
R &= \langle\psi^{\dagger}, q_{ex}\rangle
\end{align*}
\end{itemize}
%
So what does that mean? \\
It means that the response is the phase-space integral of the adjoint-weighted source distribution. 
\[
R = \int_V dV \int dE \int_{4\pi} d\vOmega\: q_{ex} \psi(\rvec_d, \vOmega, E)=\psi^{\dagger}(\rvec_0, \vOmega_0, E_0)
\]
What does that mean? \\
The weighting of the adjoint flux directly corresponds to how influential a given source particle will be on the answer we're looking for (what we defined as the adjoint source).

The importance provides the detector response to particles produced at $(\rvec_0, \vOmega_0, E_0)$.

We also talked through how looking at the differences in the equations we can see this as well. 


\end{document}
