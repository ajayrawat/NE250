\documentclass[12pt]{article}
\usepackage[top=1in, bottom=1in, left=1in, right=1in]{geometry}

\usepackage{setspace}
\onehalfspacing

\newif\ifeqns
\eqnstrue

\usepackage{amssymb}
%% The amsthm package provides extended theorem environments
\usepackage{amsthm}
\usepackage{epsfig}
\usepackage{times}
\renewcommand{\ttdefault}{cmtt}
\usepackage{amsmath}
\usepackage{graphicx} % for graphics files

% Draw figures yourself
\usepackage{tikz} 

% writing elements
\usepackage{mhchem}

% The float package HAS to load before hyperref
\usepackage{float} % for psuedocode formatting
\usepackage{xspace}

% from Denovo Methods Manual
\usepackage{mathrsfs}
\usepackage[mathcal]{euscript}
\usepackage{color}
\usepackage{array}

\usepackage[pdftex]{hyperref}
\usepackage[parfill]{parskip}

% math syntax
\newcommand{\nth}{n\ensuremath{^{\text{th}}} }
\newcommand{\ve}[1]{\ensuremath{\mathbf{#1}}}
\newcommand{\Macro}{\ensuremath{\Sigma}}
\newcommand{\rvec}{\ensuremath{\vec{r}}}
\newcommand{\vecr}{\ensuremath{\vec{r}}}
\newcommand{\omvec}{\ensuremath{\hat{\Omega}}}
\newcommand{\vOmega}{\ensuremath{\hat{\Omega}}}
\newcommand{\sigs}{\ensuremath{\Sigma_s(\rvec,E'\rightarrow E,\omvec'\rightarrow\omvec)}}
\newcommand{\el}{\ensuremath{\ell}}
\newcommand{\sigso}{\ensuremath{\Sigma_{s,0}}}
\newcommand{\sigsi}{\ensuremath{\Sigma_{s,1}}}
\newcommand{\cc}[1]{\ensuremath{\overline{#1}}}
\newcommand{\ccm}[1]{\ensuremath{\overline{\mathbf{#1}}}}
%---------------------------------------------------------------------------
%---------------------------------------------------------------------------
\begin{document}
\begin{center}
{\bf NE 250, F17\\
October 19, 2017 
}
\end{center}


\subsection*{Reflected Slab}
Duderstadt and Hamilton Chp.\ 5.III.E 

Before looking at MOC, we had looked at the criticality state for bare geometries using the one-group, steady state diffusion equation with fission
\begin{equation}
 - \nabla \cdot D(\vec{r}) \nabla	\phi(\vec{r}) + \Sigma_a(\vec{r}) \phi(\vec{r}) = \nu\Sigma_f (\vec{r})\phi(\vec{r})\:.
\end{equation}
% 
We're going to take a moment to look at what happens when we add a reflector (in your homework you're getting to look at a case with moderator). 

We'll look at a symmetric slab of fuel with thickness $a$ centered at zero, having a reflector with thickness $b$ on each side. We will have equations in two regions, recognizing symmetry (c = core and r = reflector):
\begin{align*}
-D_c\frac{d^2 \psi_c}{dx^2} &+ (\Sigma_a^c - \nu \Sigma_f^c) \psi^c(x) = 0  \qquad 0 \leq x \leq a/2\\
-D^r\frac{d^2 \psi^r}{dx^2} &+ \Sigma_a^r \psi^r(x) = 0  \qquad a/2 \leq x \leq a/2+b
\end{align*}
with boundary conditions
\begin{align*}
\frac{d \phi^c}{dx}|_{x=0} &= 0 \\
\phi^c(a/2) &= \phi^r(a/2)\\
J^c(a/2) &= J^r(a/2)\\
\phi^r(a/2+\tilde{b}) &= 0\:.
\end{align*}

We anticipate that we will need an exact balance of materials and geometry to get criticality. Let's see what that relationship turns out to be. 
From previous experience and using symmetry, we know (or can solve for) the core region solution will look like:
\begin{align*}
\phi^c(x) &= C_1 \cos(B_m^c x)\\
\text{and }&(B_m^c)^2 = \frac{\nu \Sigma_f^c - \Sigma_a^c}{D^c}
\end{align*}

In the reflector, we know 
\begin{align*}
\phi^r(x) &= C_3 \sinh(\frac{a/2+\tilde{b} - x}{L^r}) + C_4 \cosh(\frac{a/2+\tilde{b} - x}{L^r}) \\
C_4 &= 0 \text{ from the vacuum condition.}
\end{align*}

Now we need to apply the rest of the interface conditions:
\begin{align*}
C_1 \cos(\frac{B_m^c a}{2}) &= C_3 \sinh(\frac{\tilde{b}}{L^r})\\
C_1 D^c B_m^c \sin(\frac{B_m^c a}{2}) &= C_3 \frac{D^r}{L^r}\cosh(\frac{\tilde{b}}{L^r})
\end{align*}
We can manipulate these to see
\[
D^c B_m^c \tan(\frac{B_m^c a}{2}) = \frac{D^r}{L^r}\coth(\frac{\tilde{b}}{L^r})\:.
\]
This is the criticality relationship for a reflected slab. You can see it relates the materials ($B_m$ in the core and $L$ in the reflector) and the geometry ($a$ and $b$). These all have to match for criticality.

Duderstadt discusses some numerical and graphical approaches for solving this problem. In the graphical approach it can be seen that
\begin{align*}
\frac{B_m^c a}{2} &< \frac{\pi}{2}\\
(B_m^c)^2 &< (\frac{\pi}{a})^2
\end{align*}

Recall that in an \textit{unreflected core} $B_m = \frac{\pi}{\tilde{a}}$. This means that the width for a reflected core is smaller than the width for a bare core. The difference is what we call \textbf{reflector savings}:
\[
\delta = a_{\text{bare}} - a_{\text{reflected}}.
\]

\subsection*{Eigenvalue Solution}
Some from Duderstadt and Hamilton Chp.\ 5.IV (see textbook for rest of notes)

Transcendental solutions and graphing techniques are all well and good, but not idea. We'd rather have something that we can solve much more easily. That's where eigenvalue solution techniques come in. 




\subsection*{In the Transport Equation}
We can do the same kind of thing in the transport equation when there is fission and no source:
\begin{align*}
\bigl[\vOmega \cdot \nabla + \Sigma_t\bigr] \psi(\vec{r}, E, \vOmega) &= \int_{4 \pi} d\vOmega' \int_0^{\infty} dE' \: \Sigma_s(E', \vOmega' \rightarrow E, \vOmega) \psi(\vec{r}, E', \vOmega')\\
 &+ \frac{\chi(E)}{4 \pi}\int_0^{\infty} dE' \: \nu(E') \Sigma_f(E') \int_{4 \pi} d\vOmega' \:\psi(\vec{r}, E', \vOmega')
\end{align*}
%
We again use a mathematical ``knob" to give us more flexibility. However, with the TE there are two ways we do this:
\begin{enumerate}
\item Alter the effective cross section by adding an $\alpha$  term (which I don't think anyone does in diffusion), or
\item Alter the effective fission yield, $\nu$, by scaling it with $k$ (which is what we just did with diffusion).
\end{enumerate}

\textbf{The $\alpha$ version}, which is used more frequently at LLNL and LANL, looks like this:
%
\begin{align*}
\bigl[\vOmega \cdot \nabla + \bigl(\Sigma_t + \underbrace{\frac{\alpha_0}{v}}_{\text{new}}\bigr)\bigr]\psi(\vec{r}, E, \vOmega) &= \int_{4 \pi} d\vOmega' \int_0^{\infty} dE' \: \Sigma_s(E', \vOmega' \rightarrow E, \vOmega) \psi(\vec{r}, E', \vOmega')\\
 +& \frac{\chi(E)}{4 \pi}\int_0^{\infty} dE' \: \nu(E') \Sigma_f(E') \int_{4 \pi} d\vOmega' \:\psi(\vec{r}, E', \vOmega')
\end{align*}
+ boundary conditions.\\
%
Some important notes about this
\begin{itemize}
\item In the $\alpha$-evaluation problem, the total cross section is modified by a 1/v absorber.
\item There is numerical difficulty if $\alpha_0 < 0$.\\
If $\frac{-\alpha_0}{v} > \Sigma_t$ then total interaction becomes a ``source" rather than a loss!
\item If $\alpha_0 > 0$, absorption of \textbf{slow} neutrons is enhanced by an $\alpha_0$/v absorber $\rightarrow$ harder spectrum.
\item If $\alpha_0 < 0$, absorption of \textbf{fast} neutrons is enhanced by an $\alpha_0$/v absorber $\rightarrow$ softer spectrum.
\item These spectral effects are important because reaction rate is $\int_0^{\infty} dE \:\Sigma_j(E) \phi(E)$
\end{itemize}

\textbf{The $k$-eigenvalue problem} is a little bit different:
%
\begin{align*}
\bigl[\vOmega \cdot \nabla + \Sigma_t \bigr]\psi(\vec{r}, E, \vOmega) &= \int_{4 \pi} d\vOmega' \int_0^{\infty} dE' \: \Sigma_s(E', \vOmega' \rightarrow E, \vOmega) \psi(\vec{r}, E', \vOmega')\\
 +& \underbrace{\frac{1}{k}}_{\text{new}}\frac{\chi(E)}{4 \pi}\int_0^{\infty} dE' \: \nu(E') \Sigma_f(E') \int_{4 \pi} d\vOmega' \:\psi(\vec{r}, E', \vOmega')
\end{align*}
\vspace*{-.5em}
+  boundary conditions.\\
%
Important notes about this version:
\begin{itemize}
\item we can see that $k=1$ means $\alpha_0 = 0$ and vice versa.
\item $k$ as a multiplication factor: the ratio of neutron production in one generation to the neutron production in the previous generation.
\item We often use the $k$ version because we don't have the mathematical difficulties associated with the $\alpha$ form, and the physical interpretation is helpful.
\end{itemize}

More on the physical meaning: Let's define $p$ as all phase space, then
  \[\int dp = \int_{V} d\rvec \int_{4 \pi} d\vOmega \int_0^{\infty} dE\]
compute neutron production in $p$:
  \[\int dp\:\frac{\chi(E)}{4 \pi}\int_0^{\infty} dE' \: \nu(E') \Sigma_f(E') \int_{4 \pi} d\vOmega' \:\psi(\vec{r}, E', \vOmega') \]
compute neutron losses in $p$:
  \[\int dp\:\biggl[\vOmega \cdot \nabla \psi + \Sigma_t \psi - \int_{4 \pi} d\vOmega' \int_0^{\infty} dE' \: \Sigma_s(E', \vOmega' \rightarrow E, \vOmega) \psi(\vec{r}, E', \vOmega') \biggr] \]
Steady state: losses in this generation = production in the immediate past generation.\\
$k$ is the ratio of these two equations, giving the multiplication factor over $p$.\\
Note that $p$ is arbitrary, it applies pointwise over the system (limit as $V$ goes to zero).\\
We move the scattering to the LHS since it involves neutrons from this generation while fission is the only term involving neutrons from the previous one.

\end{document}