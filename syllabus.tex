%% LyX 2.1.3 created this file.  For more info, see http://www.lyx.org/.
%% Do not edit unless you really know what you are doing.
\documentclass[english]{article}
\usepackage[T1]{fontenc}
\usepackage[latin9]{inputenc}
\usepackage{geometry}
\geometry{verbose,tmargin=1.5in,bmargin=1.5in,lmargin=1.25in,rmargin=1.25in}
\usepackage{fancyhdr}
\pagestyle{fancy}
\usepackage{color}
\usepackage{setspace}
\usepackage{hyperref}

\makeatletter
\AtBeginDocument{
  \def\labelitemi{\normalfont\bfseries{--}}
}

\makeatother

\usepackage{babel}
\begin{document}
\date{\vspace{-10ex}}
\fancyhead{}
\fancyfoot{}
\fancyhead[LO,LE]{\footnotesize{NE 250 - Fall `15}}
\fancyhead[RO,RE]{\footnotesize{Syllabus}}
\fancyfoot[CO,CE]{\thepage}
\renewcommand{\headrulewidth}{0.4pt}
\renewcommand{\footrulewidth}{0.4pt}
% To change color redefine one of the existing one
% \definecolor{red}{RGB}{176, 23, 31}
% Modify to use garamond --> added to the preamble in "Document">"Settings"
% \usepackage[T1]{fontenc}
% \usepackage{lmodern}
% \usepackage{garamond}


\title{\textsc{\textcolor{black}{\rule[0.5ex]{0.75\columnwidth}{1pt}}}\textbf{\textsc{\textcolor{red}{}}}\\
\textbf{\textsc{\textcolor{blue}{Nuclear Engineering 250}}}\textbf{\textsc{\textcolor{red}{}}}\\
\textsc{\textcolor{black}{\rule[0.5ex]{0.2\columnwidth}{0.5pt}}}\textbf{\textsc{\textcolor{red}{}}}\\
\textbf{\textsc{\textcolor{blue}{Nuclear Reactor Theory}}}\textbf{\textsc{\textcolor{red}{}}}\\
\textsc{\textcolor{black}{\rule[0.5ex]{0.75\columnwidth}{1pt}}}\textbf{\textsc{\textcolor{red}{}}}\\
}

\maketitle
\begin{center}
Fall 2017
\par\end{center}

\begin{center}
Th 1:00 -- 3:00 PM\\
F  3:00 -- 5:00 PM
\par\end{center}

\begin{center}
205 Wheeler\bigskip{}
\thispagestyle{empty}
\par\end{center}


\subsection*{\noindent \textsc{\textcolor{blue}{Course Objectives}}}

\noindent The objectives of this course are to provide graduate students
in Nuclear Engineering with a deep understanding of physical principles
of reactor design and analysis.


\subsection*{\noindent \textsc{\textcolor{blue}{Instructor}}}

\noindent \href{https://www.nuc.berkeley.edu/people/rachel-slaybaugh}{Rachel Slaybaugh}\\
\href{slaybaugh@berkeley.edu}{slaybaugh@berkeley.edu}\\
4173 Etcheverry Hall\\
Office hours Fridays, 10:00 -- 11:00  AM or by appointment

\subsection*{\noindent \textsc{\textcolor{blue}{Reader}}}

%\noindent Kelly Rowland\\
%\href{krowland@berkeley.edu}{krowland@berkeley.edu}\\
%3115 Etcheverry Hall\\
%Office hours TBD


\subsection*{\noindent \textsc{\textcolor{blue}{Textbook}}}

\noindent J.J. Duderstadt and L.J. Hamilton. \emph{Nuclear Reactor
Analysis.} John Wiley \& Sons (1976), ISBN 0-471-22363-8

%\noindent K.O. Ott and W.A. Bezella. \emph{Introductory Nuclear Reactor
%Statics.} American Nuclear Society (1989), ISBN 0-89448-033-2


\subsection*{\noindent \textsc{\textcolor{blue}{Additional Resources}}}

\begin{itemize}
\item Class GitHub: \href{https://github.com/rachelslaybaugh/NE250}{https://github.com/rachelslaybaugh/NE250}

\item Bcourses: \href{https://bcourses.berkeley.edu}{https://bcourses.berkeley.edu}

\item Library: \href{http://www.lib.berkeley.edu/node}{http://www.lib.berkeley.edu/node}

\item KAERI: \href{http://atom.kaeri.re.kr/}{http://atom.kaeri.re.kr/}
\item U.S.\ Nuclear Data\href{http://www.nndc.bnl.gov/}{http://www.nndc.bnl.gov/}

\item Serpent official website: \href{http://montecarlo.vtt.fi/}{http://montecarlo.vtt.fi/} 

\item DECF (1171 and 1111 Etcheverry): \href{http://www.decf.berkeley.edu/}{http://www.decf.berkeley.edu/}

\item Free Python ebooks:
\href{http://www.leettips.org/2013/02/top-10-free-python-pdf-ebooks-download.html}{http://www.leettips.org/2013/02/top-10-free-python-pdf-ebooks-download.html}

\item The Hacker Within: \href{http://thehackerwithin.github.io/berkeley/}{http://thehackerwithin.github.io/berkeley/}

\item Software Carpentry: \href{http://software-carpentry.org/lessons.html}{http://software-carpentry.org/lessons.html}
\end{itemize}


\subsection*{\noindent \textsc{\textcolor{blue}{Prerequisites}}}

\noindent Students are expected to be familiar with the following
topics:
\begin{itemize}
\item basic nuclear physics (i.e. interactions of radiation with matter,
cross-sections, reaction rates, fission chain reaction, multiplication
factor, four- and six-factor formula);
\item \noindent solution of linear, first, and second order differential
equations;
\item \noindent vector calculus and special functions (Bessel functions, exponential
integrals); and
\item basic programming and unix knowledge.
\end{itemize}

\subsection*{\noindent \textsc{\textcolor{blue}{Course Policies}}}

\noindent Attendance is mandatory. Request approval for absence for extenuating circumstances prior to absence. \\

\noindent Take-home tests (homework) will be posted on bCourses biweekly.
Assignments will require a considerable amount of time (don't start
working on it the night before!) and no extension will be granted.
The lowest grade will be dropped (play it well!).\\

\noindent On an irregular schedule, but with a two weeks notice, a
research paper will be assigned for reading. A student will be selected
every week to prepare a critique of the paper based on an extensive
literature review. The student will present their critique and a class
discussion will follow. More details to be provided.

\noindent Grade proportion:
\begin{itemize}
\item \noindent Take-home tests (6): 60\% (lowest grade is dropped)
\item \noindent Paper critique: 30\%
\item \noindent In class participation: 10\%
\end{itemize}
\begin{singlespace}
\noindent Grading scale (tentative): A+ > 95\%, A > 91\%, A- > 87\%,
B+ > 83\%, B > 79\%, B- > 75\%, C+ > 71\%, C > 67\%, C- > 63\%, D+
> 59\%, D > 55\%, D- > 50\%, F $\leq$ 50\%.
\end{singlespace}


\subsection*{\textsc{\textcolor{blue}{Classroom Decorum}}}

This class leverages and thrives on the active participation of the
students; therefore, it is preferable (meaning mandatory) that students
turn off all electronic devices (laptop, tablets, cellphones,
etc.) during class. Exceptions may be granted for note taking.


\subsection*{\textsc{\textcolor{blue}{Academic}}\textcolor{blue}{{} }\textsc{\textcolor{blue}{Integrity}}}

\noindent \textbf{The student community at UC Berkeley has adopted
the following Honor Code:} \textquotedblleft As a member of the UC
Berkeley community, I act with honesty, integrity, and respect for
others.\textquotedblright{} The hope and expectation is that you will
adhere to this code.\\

\noindent \textbf{Collaboration and Independence:} Reviewing lecture
and reading materials and studying for exams can be enjoyable and
enriching things to do with fellow students. This is recommended.
However, unless otherwise instructed, homework assignments are to
be completed independently and materials submitted as homework should
be the result of one\textquoteright s own independent work.\\

\noindent \textbf{Cheating:} A good lifetime strategy is always to
act in such a way that no one would ever imagine that you would even
consider cheating. Anyone caught cheating on a quiz or exam in this
course will receive a failing grade in the course and will also be
reported to the University Center for Student Conduct. To
guarantee that you are not suspected of cheating, please keep your
eyes on your own materials and do not converse with others during
the quizzes and exams.\\

\noindent \textbf{Plagiarism:} To copy text or ideas from another
source without appropriate reference is plagiarism and will result
in a failing grade for your assignment and usually further disciplinary
action. For additional information on plagiarism and how to avoid
it, see, for example:\\

\noindent \href{http://www.lib.berkeley.edu/instruct/guides/citations.html#Plagiarism}{http://www.lib.berkeley.edu/instruct/guides/citations.html\#Plagiarism}

\noindent \href{http://gsi.berkeley.edu/teachingguide/misconduct/prevent-plag.html}{http://gsi.berkeley.edu/teachingguide/misconduct/prevent-plag.html}\\

\noindent \textbf{Academic Integrity and Ethics:} Cheating on exams
and plagiarism are two common examples of dishonest, unethical behavior.
Honesty and integrity are of great importance in all facets of life.
They help to build a sense of self-confidence and are key to building
trust within relationships, whether personal or professional. There
is no tolerance for dishonesty in the academic world, for it undermines
what we are dedicated to doing\textendash{}furthering knowledge
for the benefit of humanity. Your experience as a student at UC Berkeley
is hopefully fueled by passion for learning and replete with fulfilling
activities. We do appreciate that being a student can be stressful.
There may be times when there is temptation to engage in some kind
of cheating in order to improve a grade or otherwise advance your
career. This could be as blatant as having someone else sit for you
in an exam, or submitting a written assignment that has been copied
from another source. It could be as subtle as glancing at a fellow
student\textquoteright s exam when you are unsure of an answer to
a question and are looking for some confirmation. One might do any
of these things and potentially not get caught. However, if you cheat,
no matter how much you may have learned in this class, you have failed
to learn perhaps the most important lesson of all.


\subsection*{\textsc{\textcolor{blue}{Accessibility}}}

\noindent Please see me as soon as possible if you need particular
accommodations, and we will work out the necessary arrangements.


\subsection*{\textsc{\textcolor{blue}{Contents}}}
We will cover these topics (subject to change):
\begin{itemize}
\item Fundamental neutron physics concepts

\begin{itemize}
\item nuclear reactions
\item cross-sections
\item reaction rates
\item fission and chain reaction
\item multiplication factor
\item four- and six-factor formula
\end{itemize}
\item \noindent The neutron transport equation as static balance
\item The diffusion equation and separation of space and energy dependence
\item The space-dependent neutron flux

\begin{itemize}
\item flux shape in homogeneous regions
\item flux shape in slab, sphere, and cylindrical bare core
\item flux shape in multi-region cores
\end{itemize}
\item The energy dependent neutron flux; multi-group diffusion
\item Neutron slowing-down (fast spectrum calculations)

\begin{itemize}
\item slowing down in hydrogen
\item slowing down in non-hydrogenous materials
\item slowing down with resonance absorption
\end{itemize}
\item Thermal spectrum calculations
\item Space and energy dependence in cell calculations
\item Solving the neutron transport problem

\begin{itemize}
\item the discretization approach: $S_{n}$ theory
\item the spherical harmonic approach: $P_{n}$ theory
\end{itemize}
\item Perturbation theory and sensitivity analysis
\item The adjoint transport equation
\item Neutronics modeling tools: Serpent
\end{itemize}

\subsection*{}
\end{document}
